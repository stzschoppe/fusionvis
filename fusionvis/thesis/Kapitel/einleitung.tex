%----------------------------------------------------------------%
% einleitung.tex																					       %
%----------------------------------------------------------------%
\chapter{Einleitung}

\section{Motivation}
Wir leben in einer Zeit stetig voranschreitender Technologisierung. Dies �u�ert sich f�r jeden sichtbar auf vielerlei Art und Weise. Massenspeicher mit vor Jahren noch unvorstellbaren Kapazit�ten, stetig wachsende Prozessorleistung und massiver Fortschritt in der Daten�bertragung sind hier nur als Beispiele zu nennen. Solche Entwicklungen sind es, die das Sammeln, Verarbeiten und Speichern von riesigen Datenmengen erst erm�glichen.  Diesen Fortschritt gilt es zu nutzen und auf m�gliche Anwendungsfelder auszuweiten. 

\label{motivation}
Betrachtet man ein Schlachtfeld, sei es im Rahmen einer �bung, einer kriegerischen Auseinandersetzung oder eines Konflikts, so werden auch hier Informationen gesammelt und ausgewertet. Man stelle sich folgende Situation vor: Drei eigene Panzer bewegen sich durch das Gel�nde. Pl�tzlich kl�ren sie zwei feindliche Fahrzeuge auf. Jeder einzelne eigene Panzer setzt eine Meldung ab und beschreibt, was er sieht. Dies f�hrt zu sechs Meldungen. Der S2\footnote{Der S2-Offizier ist verantwortlich f�r die milit�rische Sicherheit, milit�risches Nachrichtenwesen mit Aufkl�rung und Zielfindung, elektronisch Kampff�hrung und eben die Wehrlage}-Offizier muss nun aus diesen Meldungen ein Lagebild erstellen. Dabei sind aus der vorhandenen (teilweise redundanten) Information zwei statt sechs feindliche Fahrzeuge zu extrahieren. 

F�r das geschilderte Beispiel scheint es nicht notwendig, diese Informationsauswertung zu automatisieren. In der Realit�t hingegen  erfordert es viel Zeit, diese Arbeit zu erledigen\footnote{Ein Panzerbataillon der Bundeswehr umfasst zum Beispiel ca. 40 Kampfpanzer}. Und genau dies ist ein gro�es Problem, denn je �lter die Lageinformation ist, umso weniger aussagekr�ftig ist sie. Entscheidungen, die darauf basierend getroffen werden, k�nnen falsch oder unverh�ltnism��ig sein. Um diesen Missstand zu beseitigen, gilt es, den S2-Offizier bei seiner Arbeit technisch zu unterst�tzen.

\section{Ziel der Arbeit}

Eine technische Unterst�tzung kann in unterschiedlichen Abstufungen geschehen. So k�nnen die separaten Meldungen zu einer gro�en Meldung zusammengefasst werden. Dies ist aber nicht sonderlich hilfreich, wenn zum Beispiel aus vielen einzelnen Datens�tzen einfach eine zusammenh�ngende Datensammlung erstellt wird. Denn wenn diese in einem textuellen Format vorliegt, ist sie von einem Menschen nur schwer zu verstehen. Weiterhin k�nnen die auftretenden Redundanzen nicht einfach erfasst, geschweige denn �berhaupt genutzt werden.

Aus diesem Grund m�chte ich mich in dieser Arbeit mit Visualisierungsformen auseinandersetzen, die eine menschenlesbare Sicht auf multisensorische Daten bieten. Zuerst sollen Visualisierungsm�glichkeiten aufgezeigt werden. Bei unterschiedlichen Ans�tzen sollen diese bewertet werden.

Ergebnis dieser Betrachtungen wird ein Framework zur Darstellung multisensorischer Daten. Dieses verwende ich dann prototypisch dazu, die eingangs erw�hnte Problemstellung des S2-Offiziers zu bearbeiten und ihm eine, f�r seine Lageerstellung hilfreiche, Sicht auf die eingehenden Meldungen zu geben.

Die Erweiterbarkeit des Frameworks soll durch dessen Verwendung zur Darstellung von NDP [ABK] Daten gezeigt werden.


\section{Aufbau der Arbeit}