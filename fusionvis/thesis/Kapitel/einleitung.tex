%----------------------------------------------------------------%
% einleitung.tex																					       %
%----------------------------------------------------------------%
\chapter{Einleitung}

\section{Motivation}
Wir leben in einer Zeit stetig voranschreitender Technologisierung. Dies �u�ert sich f�r jeden sichtbar auf vielerlei Art und Weise. Massenspeicher mit vor Jahren noch unvorstellbaren Kapazit�ten, stetig wachsende Prozessorleistung und massiver Fortschritt in der Daten�bertragung sind hier nur als Beispiele zu nennen. Solche Entwicklungen sind es, die das Sammeln, Verarbeiten und Speichern von riesigen Datenmengen erst erm�glichen.  Diesen Fortschritt gilt es zu nutzen und auf m�gliche Anwendungsfelder auszuweiten. \\
Betrachtet man ein Schlachtfeld, sei es im Rahmen einer �bung, einer kriegerischen Auseinandersetzung oder eines Konflikts, so werden auch hier Informationen gesammelt und ausgewertet. Man stelle sich folgende Situation vor: Drei eigene Panzer bewegen sich durch das Gel�nde. Pl�tzlich kl�ren sie zwei feindliche Fahrzeuge auf. Jeder einzelne eigene Panzer setzt eine Meldung ab und beschreibt, was er sieht. Dies f�hrt zu sechs Meldungen. Der S2\footnote{Der S2-Offizier ist verantwortlich f�r die Milit�rische Sicherheit, Milit�risches Nachrichtenwesen mit Aufkl�rung und Zielfindung, elektronisch Kampff�hrung und eben die Wehrlage}-Offizier muss nun aus diesen Meldungen ein Lagebild erstellen. Dabei gilt es aus der vorhandenen (teilweise redundanten) Information die zwei statt sechs feindliche Fahrzeuge zu erkennen. \\
F�r das geschilderte Beispiel scheint es nicht notwendig, diese Informationsauswertung zu automatisieren. In der Realit�t hingegen \footnote{Ein Panzerbataillon der Bundeswehr umfasst zum Beispiel ungef�hr 40 Kampfpanzer} erfordert es viel Zeit, diese Arbeit zu erledigen. Und genau dies ist ein gro�es Problem, denn je �lter die Lageinformation ist, umso weniger aussagekr�ftig ist sie. Entscheidungen, die darauf basierend getroffen werden, k�nnen daraufhin falsch oder unverh�ltnism��ig sein. Um diesen Missstand zu beseitigen, gilt es, den S2-Offizier bei seiner Arbeit technisch zu unterst�tzen.

\section{Ziel der Arbeit}

\section{Aufbau der Arbeit}