%----------------------------------------------------------------%
% theorie.tex																			       %
%----------------------------------------------------------------%

\chapter{Multisensorische Daten und ihre Visualisierung}
\label{ch:theorie}
\section{Theoretische Einordnung}
\label{sec:TheoretischeEinordnung}

\subsection{Eigenschaften von multisensorischen Daten}
\label{sec:EigenschaftenVonMultisensorischenDaten}
 
\subsection{Multisensor-Fusion}
\label{sec:MultisensorFusion}
\cite{citeulike:4775427} \cite{citeulike:894117} \cite{Hall2004fusion}
\section{Visualisierungsm�glichkeiten}
\label{sec:Visualisierungsmoeglichkeiten}
 
\subsection{Darstellung der Dimensionen}
\label{sec:DarstellungDerDimensionen}

\subsubsection{Dreidimensionale Darstellung}
\label{sec:DreidimensionaleDarstellung}

\subsubsection{Vierdimensionale Darstellung}
\label{sec:VierdimensionaleDarstellung}

\subsubsection{Niederdimensionale  Ans�tze}
\label{sec:NiederdimensionaleAnsaetze}

\subsection{Darstellung der Datens�tze}
\label{ch:visualisierungDatum}

\subsubsection{Formen}
\label{sec:Formen}

\subsubsection{Visualisierung von Eigenschaften}
\label{sec:VisualisierungVonEigenschaften}

\subsection{Informationsgewinnung durch Visualisierung}
\label{sec:InformationsgewinnungDurchVisualisierung}

\subsubsection{Bewegungskegel}
\label{sec:TheorieBewegungskegel}

\subsubsection{Zugeh�rigkeitslinien}
\label{sec:Zugehoerigkeitslinien}
