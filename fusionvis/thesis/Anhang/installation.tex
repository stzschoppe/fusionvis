\chapter{Installationsanleitungen}
An dieser Stelle wird erl�utert, wie die mitgelieferte Software zu installieren und zu konfigurieren ist. Erfahrene Benutzer
k�nnen diesen Teil �berspringen und die Software nach ihren Vorlieben installieren. Dieser Teil dient nur als Anhalt und hat
nicht den Anspruch einer vollst�ndigen Anleitung. Vielmehr werden die wichtigsten Schritte erl�utert und eine M�glichkeit zur
Konfiguration aufgezeigt.

Die entsprechenden Programme auf der beiliegenden CD sind f�r ein Windowsbetriebssystem ausgelegt. Das Testsystem war dabei
Windows XP SP 2 32-bit. Die Software ist auch unter anderen Betriebssystemen erh�ltlich. F�r eine Installationsanleitung f�r weitere
Betriebsysteme wird auf das Internet verwiesen.

\section{Java}

\begin{enumerate}
	\item F�hren Sie die Datei \emph{jdk-6u16-windows-i586.exe} aus.
	\item Sollte sich eine Sicherheitswarnung �ffnen, w�hlen Sie "`Ausf�hren"'.
	\item Folgen Sie den Anweisungen des Installationsdialogs.
	\item F�gen Sie nach der Installation die Umgebungsvariable \emph{JAVA\_HOME} hinzu, falls diese noch nicht existiert: \\
		Start $\Longrightarrow$ Systemeinstellungen $\Longrightarrow$ System $\Longrightarrow$ Erweitert $\Longrightarrow$ Umgebungsvariablen 
		$\Longrightarrow$ Systemvariablen $\Longrightarrow$ Neu \\
		Der Inhalt der Variable ist das Installationsverzeichnis des JDK. Nach der Ver�nderung muss die Konsole neu gestartet werden.
	\item F�gen Sie das Verzeichnis "`\$JAVA\_HOME\$/bin"' der PATH Variable hinzu.
	\item Falls gew�nscht, kann nun noch die Datei \emph{jre-6u16-windows-i586.exe} ausgef�hrt werden.
	\item Der Installationsverlauf ist analog, jedoch ist keine weitere Variable n�tig.
	\item Starten Sie Ihr System neu.
\end{enumerate}

\section{Eclipse}

\begin{enumerate}
	\item Entpacken Sie die Datei \emph{eclipse-rcp-galileo-win32.zip} mit einem Programm Ihrer Wahl in ein Verzeichnis Ihrer Wahl.
	\item F�hren Sie die \emph{eclipse.exe} im Verzeichnis "`eclipse"' aus.
	\item Definieren Sie einen Ordner als Ihren Workspace.
	\item Installieren Sie das EclipseCORBA Plug-in: \\
		Help $\Longrightarrow$ Install New Software ... $\Longrightarrow$ Add \\
		Name: EclipseCORBA \\ 
		Location: http://eclipsecorba.sf.net/update \\
		W�hlen Sie alle Items im Men� aus und folgen Sie dem Dialogverlauf.
\end{enumerate}

\section{JacORB}

\begin{enumerate}
	\item Entpacken Sie die Datei \emph{jacorb-2.3.1-src.zip} mit einem Programm Ihrer Wahl in ein Verzeichnis Ihrer Wahl.
	\item F�gen Sie die Variable \emph{JACORB\_HOME} zu den Umgebungsvariablen des Systems hinzu.
	\item F�gen Sie das Verzeichnis "`\$JACORB\_HOME\$/bin"' der PATH Variable hinzu.
	\item �ffnen Sie die Datei \emph{orb.properties} im Verzeichnis "`\$JACORB\_HOME\$/etc"' mit einem Programm Ihrer Wahl.
	\item �ndern Sie die Variable \emph{jacorb.config.dir} auf "`\$JACORB\_HOME\$/etc"' ab.
	\item �ndern Sie die Variable \emph{ORBinitRef.NameService} ab, auf einen Ordner Ihrer Wahl mit Schreibrechten. Es ist m�glich
		die Voreinstellung mit "`file:/c:/NS\_Ref"' beizubehalten.
	\item �ndern Sie die Variable \emph{jacorb.naming.ior\_filename} ab, auf den selben Ordner wie \emph{ORBinitRef.NameService}.
	\item Kopieren Sie die ge�nderte \emph{orb.properties} in Ihr Benutzerverzeichnis, in "`\$JAVA\_HOME\$/lib"' und in "`\$JAVA\_HOME\$/jre6/lib"'
	\item Ersetzen Sie beim Kopiervorgang eine eventuell bereits vorhandene \emph{orb.properties}.
	\item �ffnen Sie die Datei \emph{jacorb.properties} im Verzeichnis "`\$JACORB\_HOME\$/etc"' mit einem Programm Ihrer Wahl.
	\item �ndern Sie analog die Variable \emph{jacorb.naming.ior\_filename} und die Variable \emph{ORBinitRef.NameService}.
	\item Testen Sie die Einstellungen, indem Sie die Konsole aufrufen. Bei Eingabe des Kommandos ns, sollte der NameServer von JacORB starten.	
	\item Bei weitergehender Konfiguration konsultieren Sie die Datei ProgrammingGuide.pdf im Verzeichnis "`\$JACORB\_HOME\$/doc"'
\end{enumerate}

\section{OpenORB}

\begin{enumerate}
	\item Kopieren Sie aus dem Verzeichnis "`CD:/OpenORB"' alle Ordner bis auf den Ordner \emph{src} in ein Verzeichnis Ihrer Wahl.
	\item F�gen Sie die Variable \emph{TCOO\_HOME} zu den Umgebungsvariablen des Systems hinzu. Diese enth�lt das Verzeichnis, in das Sie
		die Ordner kopiert haben.
	\item �ffnen Sie im Verzeichnis "`\$TCOO\_HOME\$/OpenORB/config"' die Datei \emph{OpenORB.xml} mit einem Programm Ihrer Wahl.
	\item Editieren Sie die Datei, wie im unten stehenden Listing zu erkennen. 
	\item Geben Sie folgende Kommandos in die Konsole ein: \\
		cd \$TCOO\_HOME\$/tools/bin \\
		updateConfig OpenORB OpenORB.xml
	\item �ffnen Sie im Verzeichnis "`\$TCOO\_HOME\$/OpenORB/config"' die Datei \emph{trader.xml} mit einem Programm Ihrer Wahl.
	\item �ndern Sie den Namen des TradingServers in einen beliebig gew�hlten Namen um.
	\item Geben Sie folgende Kommandos in die Konsole ein: \\
		cd \$TCOO\_HOME\$/tools/bin \\
		updateConfig TradingService trader.xml
	\item Erstellen Sie eine Batchdatei mit folgendem Inhalt: \\
		cd \$TCOO\_HOME\$
		java \\
		-Xbootclasspath/p:/OpenORB/lib/endorsed/openorb\_orb\_omg-1.4.0.jar \\
		-Dopenorb.home.path=\$TCOO\_HOME\$ \\
		-jar /tools/lib/launcher.jar org.openorb.trader.Server
	\item Ab dem Kommando \emph{java} muss alles in einer Zeile stehen.
	\item Der Aufruf dieser Datei startet den Trading Server, der auf den JacORB Naming Server zugreift.
\end{enumerate}

\begin{verbatim}
...
<profile name="default" xlink:href="$(openorb.home)config/
default.xml#default">
  <property name="NameService" value="file:/c:/NS_Ref" />
</profile>
...
\end{verbatim}

\section{Managementframework}

\begin{enumerate}
	\item �ffnen Sie Eclipse.
	\item Importieren Sie das Projekt \emph{ManagementFramework} in Ihren Workspace: \\
		File $\Longrightarrow$ Import $\Longrightarrow$ Existing Projects into Workspace \\
		W�hlen Sie das Verzeichnis "`CD:/Implementierung/ManagementFramework"' als \emph{root directory}. \\
		Best�tigen Sie dies mit \emph{Finish}.
	\item Passen Sie im \emph{Build Path} des Projekts die Referenz auf die log4j-1.2.15.jar entsprechend ihres Systems an.
	\item �ffnen Sie das \emph{Run} Profil der Klasse \emph{start/starter.java}: \\
		Rechtsklick auf starter.java $\Longrightarrow$ Run As $\Longrightarrow$ Run Configurations... 
	\item Geben Sie im Tab \emph{Arguments} im Feld \emph{VM arguments} Folgendes ein: \\
		-Djava.endorsed.dirs=[PATH\_TO\_JACORB]/bin \\
		-Djacorb.home=[PATH\_TO\_JACORB]
	\item F�gen Sie im Tab \emph{Classpath} unter \emph{Bootstrap Entries} folgende externe JARs hinzu: \\
		backport-util-concurrent.jar \\
		jacorb.jar \\
		logkit-1.2.jar \\
		slf4j-api-1.5.6.jar \\
		slf4j-jdk14-1.5.6.jar \\
		Diese befinden sich im Verzeichnis "`\$JACORB\_HOME\$/lib"'.
	\item Best�tigen Sie mit \emph{Apply} und f�hren Sie die starter.java mit \emph{Run} aus.
\end{enumerate}

\section{PluginCore Source}
	
\begin{enumerate}
	\item �ffnen Sie Eclipse.
	\item Importieren Sie das Projekt \emph{de.inf06.plugin.core} in Ihren Workspace. 
	\item Passen Sie die folgenden Referenzen im Build Path des Projekts an: \\
		log4j-1.2.15.jar \\
		openorb\_trader-1.4.0.jar \\
		Letztere befindet sich im Verzeichnis "`\$TCOO\_HOME\$/TradingService/lib"'.		
\end{enumerate}

\section{DCS \& DCDR}

\begin{enumerate}
	\item �ffnen Sie Eclipse.
	\item Importieren Sie das Projekt \emph{de.inf06.dcl} in Ihren Workspace. 
	\item Passen Sie die folgenden Referenzen im Build Path des Projekts an: \\
		log4j-1.2.15.jar \\
		openorb\_trader-1.4.0.jar \\
		jacorb.jar \\
		SNMP4J.jar
	\item Passen Sie das \emph{Run} Profil der Klassen \emph{StartDCDR.java} und \emph{StartDCS.java} analog der Klasse \emph{starter.java} des
	Projekts ManagementFramework an.	
\end{enumerate}