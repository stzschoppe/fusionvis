\documentclass[handout]{beamer}
%\documentclass{beamer}
\usepackage{pgfpages}
\usepackage{eurosym}
\setbeamertemplate{footline}[frame number]
%\pgfpagesuselayout{2 on 1}[a4paper,portrait,border shrink=5mm]
\usetheme{Copenhagen} % Beamer Theme
\usecolortheme{crane_unibw} % BeamerColor Theme
%\useoutertheme[subsection=false]{smoothbars} % Beamer Outer Theme
\setbeamertemplate{blocks}[rounded][shadow=true]
%\useoutertheme{infolines}
%\useinnertheme{rectangles}
\setbeamercovered{transparent}
%\usepackage{beamerthemeshadow}
\usepackage[latin9]{inputenc}
\usepackage{graphicx}
\usepackage[british,ngerman]{babel}
\usepackage{curves}
\usepackage{epic}
\usepackage{setspace}
\usepackage{amsmath,amssymb,dsfont,stmaryrd}
\usepackage{pgfpages}
\usepackage{hyperref}
%\usepackage{xspace}
\newcommand{\C}[1]{% stellt Buchstaben als script letter \mathcal{} dar
\ensuremath{\mathcal{#1}}\xspace}

\newcommand{\D}[1]{% stellt Buchstaben als script letter \mathds{} dar
\ensuremath{\mathds{#1}}\xspace}

\newcommand{\B}[1]{% stellt Buchstaben als script letter \mathbb{}  dar
\ensuremath{\mathbb{#1}}\xspace}

\newcommand{\F}[1]{% stellt Buchstaben als Frakturschrift dar
\ensuremath{\mathfrak{#1}}\xspace}

\newcommand{\Bf}[1]{% stellt Buchstaben im Mathemodus fett  \mathbf{}  dar
\ensuremath{\mathbf{#1}}\xspace}


\newcommand{\Z}[1]{%
\ensuremath{\mathscr{#1}}\xspace}


\newcommand{\FC}[0]{% 
\ensuremath{\mathfrak{C}}\xspace}
\newcommand{\CT}[0]{% 
\ensuremath{\mathcal{T}}\xspace}
\newcommand{\CB}[0]{% 
\ensuremath{\mathcal{B}}\xspace}


\newcommand{\FO}[0]{%
\ensuremath{\mathfrak{O}}\xspace}

\newcommand{\fett}[1]{{\bf #1}}

\newcommand{\OL}[1]{%
\ensuremath{\overline{#1}}\xspace}

\definecolor{darkblue}{rgb}{0,0,.5}
\hypersetup{pdftex=true, colorlinks=true, breaklinks=true,
  linkcolor=darkblue, menucolor=darkblue,
 pagecolor=darkblue, urlcolor=darkblue}
%\pgfpagesuselayout{4 on 1}[a4paper,landscape,border shrink=5mm]

%% FrameNr setzen (links unten) Frame / FrameGes.
\setbeamertemplate{footline}{%
\begin{beamercolorbox}[wd=\paperwidth,ht=2.25ex,dp=1ex]{date in
head/foot}%
\hfill \insertframenumber{} / \inserttotalframenumber \hspace*{1ex} 
% 
\end{beamercolorbox}}%
\usepackage{xspace}
\newcommand{\C}[1]{% stellt Buchstaben als script letter \mathcal{} dar
\ensuremath{\mathcal{#1}}\xspace}

\newcommand{\D}[1]{% stellt Buchstaben als script letter \mathds{} dar
\ensuremath{\mathds{#1}}\xspace}

\newcommand{\B}[1]{% stellt Buchstaben als script letter \mathbb{}  dar
\ensuremath{\mathbb{#1}}\xspace}

\newcommand{\F}[1]{% stellt Buchstaben als Frakturschrift dar
\ensuremath{\mathfrak{#1}}\xspace}

\newcommand{\Bf}[1]{% stellt Buchstaben im Mathemodus fett  \mathbf{}  dar
\ensuremath{\mathbf{#1}}\xspace}


\newcommand{\Z}[1]{%
\ensuremath{\mathscr{#1}}\xspace}


\newcommand{\FC}[0]{% 
\ensuremath{\mathfrak{C}}\xspace}
\newcommand{\CT}[0]{% 
\ensuremath{\mathcal{T}}\xspace}
\newcommand{\CB}[0]{% 
\ensuremath{\mathcal{B}}\xspace}


\newcommand{\FO}[0]{%
\ensuremath{\mathfrak{O}}\xspace}

\newcommand{\fett}[1]{{\bf #1}}

\newcommand{\OL}[1]{%
\ensuremath{\overline{#1}}\xspace}

\begin{document}
\title{Visualisierung multisensorischer Daten zur Vorbereitung von Datenfusion}
\subtitle{am Beispiel einer milit�rischen Lage}
\author{Stephan Tzschoppe}
\institute[Universit�t der Bundeswehr M�nchen]{Institut f�r Theoretische Informatik, \\
Mathematik und Operations Research \\ 

Fakult�t f�r Informatik \\
}
%\institute[]{Institut für \\ Fakultät für Informatik\\Universität der Bundeswehr München}
\date{18.02.2010}
%%%%%Logo auf jeder Seite%%%%%%%%%%%%%%%%%%%%%%%
\pgfdeclareimage[height=2cm]{bwlogo}{Bilder/UniBwM_logo} \logo{\pgfuseimage{bwlogo}}
%%%%%%%%%%%%%%%%%%%%%%%%%%%%%%%%%%%%%%%%%%%%%%%%
%%%%%%Logo nur auf der Titelseite
%\titlegraphic{\pgfuseimage{bwlogo}}
%%%%%%%%%%%%%%%%%%%%%%%%%%
\begin{frame}
\titlepage
\end{frame}

\begin{frame}
\frametitle{Inhaltsverzeichnis}
\begin{footnotesize}
\tableofcontents
\end{footnotesize}
\end{frame}


\section{Motivation}
\subsection{Milit�rische Lage}
\begin{frame}\frametitle{Lagebeispiel}
  \begin{itemize}[<+-| alert@+>]
  \item 2 eigene Panzer (blau)
    \begin{itemize}[<+-| alert@+>]
  	  \item bewegen sich nicht
  	  \item melden Beobachtungen in konstanten Zeitabst�nden
    \end{itemize}
  
  \item 2 feindliche Panzer (rot)
  
  \begin{itemize}[<+-| alert@+>]
  	\item ein Panzer in Querfahrt
  	\item ein Panzer in in Richtung der eigenen Kr�fte
    \end{itemize}
\end{itemize}
\end{frame}

\begin{frame}\frametitle{Darstellung der Beispiellage}
\begin{figure}[h]
	\centering
		\includegraphics[width=0.80\textwidth]{Bilder/googleearth.png}
	\caption{Lage zum ersten Meldezeitpunkt}
	\label{fig:googleearth}
\end{figure}
\end{frame}

\begin{frame}\frametitle{Darstellung der Beispiellage}
\begin{figure}[h]
	\centering
		\includegraphics[width=0.80\textwidth]{Bilder/googleearth2.png}
	\caption{Lage zum zweiten Meldezeitpunkt}
	\label{fig:googleearth2}
\end{figure}
\end{frame}

\begin{frame}\frametitle{Darstellung der Beispiellage}
\begin{figure}[h]
	\centering
		\includegraphics[width=0.80\textwidth]{Bilder/googleearth3.png}
	\caption{Lage zum dritten Meldezeitpunkt}
	\label{fig:googleearth3}
\end{figure}
\end{frame}

\subsection{Repr�sentation der Daten}

\begin{frame}[fragile,allowframebreaks]
\frametitle{Darstellung der Meldungen in XML}
\lstset{%
language=XML,
basicstyle={\ttfamily, \tiny},
numbers=left,                   % where to put the line-numbers
numberstyle=\tiny,      % the size of the fonts that are used for the line-numbers
stepnumber=1
}%
\begin{lstlisting}[breaklines=true,frame=tlRB,captionpos=b,caption={Darstellung einer Meldung in XML},label=code:xmlexample]
<Situation>
  <Units>
   <Unit>
      <Name>FriendlyTank1</Name>
      <Location>
        <Lat>52.796629714678467</Lat>
        <Lon>9.89990561649954</Lon>
        <LastModified>2009-02-20T10:25:36+01:00</LastModified>
      </Location>
    </Unit>
    <Unit>
      <Name>FriendlyTank2</Name>
      <Location>
        <Lat>52.794038961891268</Lat>
        <Lon>9.9011727699025922</Lon>
        <LastModified>2009-02-20T10:25:36+01:00</LastModified>
      </Location>
    </Unit>
  </Units>
</Situation>
\end{lstlisting}
\end{frame}

\subsection{Problemstellung}

\begin{frame}\frametitle{Problemstellung}

\begin{block}{Redundante Meldungen zu einem Objekt}
\begin{itemize}[<+-| alert@+>]
    \item Sichtung unterschiedlicher Beobachter zur selben Zeit
    \item Sichtung zu unterschiedlichen Zeiten
\end{itemize}
\end{block}
\begin{itemize}[<+-| alert@+>]
    \item Redundanzen verf�lschen die Lage
    \item Weiterverarbeitung und Analyse unm�glich
    \item \textbf{Beseitigung der Redundanzen ist notwendig}
\end{itemize}
\end{frame}


\section{Theoretische Aspekte}

\subsection{Multisensorische Daten}
\begin{frame}\frametitle{Eigenschaften Multisensorischer Daten}
 \begin{columns}[t]
\begin{column}{5cm}
    \begin{block}{Vorteile}
    \begin{itemize}
        \item Robustheit
        \item R�umlich weitreichende Abdeckung
        \item Zeitlich weitreichende Abdechung
        \item Gesteigerte Informationssicherheit
        \item erh�hte Informationsdichte
    \end{itemize}
    \end{block}
\end{column}
\begin{column}{5cm}
    \begin{block}{Nachteile}
    \begin{itemize}
        \item Schwierige Zuordnungen von Meldungen zu realen Objekten
        \item Redundanzen verf�lschen die Daten
    \end{itemize}
    \end{block}
\end{column}
\end{columns}
\end{frame}


\subsection{Multisensor Data Fusion}
\begin{frame}\frametitle{Multisensor Data Fusion}
  \begin{block}{Definition}
  \begin{quote}
		"`[Fusion is] the integration of information from multiple sources to procedure specific and comprehensive unified data about an entity"' \protect\cite{fusionintroduction}
  \end{quote}
  \end{block}
  
  \begin{block}{Data Fusion Level}
  \begin{enumerate}[<+-| alert@+>]
   \item Object Assessment
   \item Situation Assessment
   \item Impact Assessment
  \end{enumerate}
  \end{block}
\end{frame}

%\subsection{nichtlineare Skalentransformation}
%\begin{frame}\frametitle{Geographische Koordinaten -- Kartesische Koordinaten}
%  \begin{itemize}[<+-| alert@+>]
%  \item  gdgdgdd
%\end{itemize}
%\end{frame}


\section{Probleml�sung}

\subsection{Problemstellung}
\begin{frame}\frametitle{Problemstellung}
  \begin{block}{Informationsfusion durch Visualisierung}
   \begin{enumerate}
   	\item ortsgleiche Sichtungen eines Objekts
   	\item multiple Sichtungen eines Objekts zu unterschiedlichen Zeiten
   \end{enumerate}
\end{block}
\end{frame}

\subsection{Visualisierung}
\begin{frame}\frametitle{Auswahl der Dimensionen}
 \begin{columns}[t]
\begin{column}{5cm}
    \begin{block}{Dimensionen}
    \begin{itemize}
        \item x-z-Ebene: L�nge/Breite
        \item y-Achse: Zeit
        \item Vernachl�ssigung der H�he
    \end{itemize}
    \end{block}
\end{column}
\begin{column}{5cm}
\begin{figure}[h]
	\centering
		\includegraphics[width=0.7\textwidth]{Bilder/Koordinatensystem.png}
	\caption{Rechtsh\protect\"{a}ndiges Koordinatensystem \cite{guide}}
	\label{fig:Koordinaten}
\end{figure}
\end{column}
\end{columns}
\end{frame}

\subsection{FusionVis - Visualisierungstool}

\begin{frame}\frametitle{Darstellung von Daten im Visualisierungstool}
\begin{center}
	\includegraphics[width=0.7\textwidth]{Bilder/screen.png}
\end{center}
 %   \begin{itemize}
%        \item Freund-/Feind-Unterscheidung
%        \item Unterscheidung beobachtet/real
%        \item Kugel als Datenpunkt
%    \end{itemize}
\end{frame}


\begin{frame}\frametitle{Features}
 \begin{columns}[t]
\begin{column}{5cm}
    \begin{block}{}
    \begin{itemize}
        \item Plattformunabh�ngigkeit
        \item XML-Import
        \item strukturierte textuelle Datendarstellung (Baum)
        \item 3D-Darstellung der Daten
        \item voreingestellte Perspektiven
    \end{itemize}
    \end{block}
\end{column}
\begin{column}{5cm}
    \begin{block}{}
    \begin{itemize}
        \item freie Navigierbarkeit im 3D-Raum
        \item Klassifizierung durch Farben
        \item Datenfilter
        \item Selektion in der 3D-Darstellung
    \end{itemize}
    \end{block}
\end{column}
\end{columns}\end{frame}

\subsection{Fusion}
\begin{frame}\frametitle{Fusion durch Visualisierung}
  \begin{block}{Informationsfusion durch Visualisierung}
   \begin{enumerate}
   	\item ortsgleiche Sichtungen eines Objekts
   	
    \begin{itemize}
	    \item Daten am selben Ort befefinden sich an der selben Stelle
	    \item Ungenauigkeiten (Jitter) k�nnen als solche erkannt werden
     \end{itemize}
   	\item multiple Sichtungen eines Objekts zu unterschiedlichen Zeiten
   	\begin{itemize}
	    \item Annahme: Beobachtete Fahrzeuge haben eine Maximalgeschwindigkeit $v_{max}$
	    \item $v_{max}$ bestimmt die maximale Reichweite in gegebener Zeit
	    \item Reichweite eine Objektes vom Datenpunkt $p_{0}$ aus kann durch ein Volumen $K$ in Form eines Kegels dargestellt werden
	    \item F�r jeden Datenpunkt $p \neq p_{0}$ gilt:\\
	          $p\notin K \rightarrow p \quad ist \quad keine \quad Beobachtung \quad von \quad p_{0}$ 
    \end{itemize}
   \end{enumerate}
\end{block}\end{frame}

\begin{frame}\frametitle{Bewegungstrichter}
\end{frame}


\section{Fazit}

\subsection{Bewertung}
\begin{frame}\frametitle{Folientitel}
  \begin{itemize}[<+-| alert@+>]
  \item  gdgdgdd
\end{itemize}
\end{frame}

\subsection{Ausblick}
\begin{frame}\frametitle{Folientitel}
  \begin{itemize}[<+-| alert@+>]
  \item  gdgdgdd
\end{itemize}
\end{frame}


\end{document}