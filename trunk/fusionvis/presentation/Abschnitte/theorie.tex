\section{Theoretische Aspekte}

\subsection{Multisensorische Daten}
\begin{frame}\frametitle{Eigenschaften}
 \begin{columns}[t]
\begin{column}{5cm}
    \begin{block}{Vorteile}
    \begin{itemize}
        \item Robustheit
        \item R�umlich weitreichende Abdeckung
        \item Zeitlich weitreichende Abdechung
        \item Gesteigerte Informationssicherheit
        \item erh�hte Informationsdichte
    \end{itemize}
    \end{block}
\end{column}
\begin{column}{5cm}
    \begin{block}{Nachteile}
    \begin{itemize}
        \item Schwierige Zuordnungen von Meldungen zu realen Objekten
        \item Redundanzen verf�lschen die Daten
    \end{itemize}
    \end{block}
\end{column}
\end{columns}
\end{frame}

\begin{frame}\frametitle{Multisensor Data Fusion}
  \begin{block}{Definition}
  \begin{quote}
		"`[Fusion is] the integration of information from multiple sources to procedure specific and comprehensive unified data about an entity"' \protect\cite{fusionintroduction}
  \end{quote}
  \end{block}
  
  \begin{block}{Data Fusion Level}
  \begin{enumerate}[<+-| alert@+>]
   \item Object Assessment
   \item Situation Assessment
   \item Impact Assessment
  \end{enumerate}
  \end{block}
\end{frame}

\subsection{nichtlineare Skalentransformation}
\begin{frame}\frametitle{Folientitel}
  \begin{itemize}[<+-| alert@+>]
  \item  gdgdgdd
\end{itemize}
\end{frame}

