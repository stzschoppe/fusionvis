%----------------------------------------------------------------%
% fazit.tex						    		  	    										       %
%----------------------------------------------------------------%
\chapter{Fazit und Ausblick}
In diesem letzten Kapitel soll eine Bewertung der im Rahmen dieser Arbeit entstandenen Implementierung vorgenommen werden. Dazu werden die zu Beginn gesetzten Ziele als Ma�stab genommen. Anschlie�end soll aufgezeigt werden, welche Aspekte in einer weiterf�hrenden Arbeit zu tun sind und wie das Erstellte eingesetzt werden kann. 

\section{Bewertung}
Ziel dieser Arbeit sollte ein plattformunabh�ngiges Framework sein, das die Visualisierung von multisensorischen Daten erm�glicht. Es sollte einer prototypischen Implementierung dienen, welche Daten eines Gefechtssimulators darstellen kann und die Redundanzen in ihnen durch Visualisierung zu beseitigen vermag.

Das resultierende Ergebnis der arbeit ist vielversprechend. Im Prinzip konnten alle Ziele umgesetzt werden. In wie weit dies geschehen ist, wird im Folgenden f�r die einzelnen Aspekte dargelegt.

\subsubsection{Framework zur Visualisierung von Multisensorischen Daten}
\label{sec:FrameworkZurVisualisierungVonMultisensorischenDaten}

Im Zuge der Implementierung  ist ein Klassenger�st entstanden, das aus drei Teilen besteht. Der erste ist eine Datenstruktur, die durch ihre modulare Struktur eine hohe Flexibilit�t und Erweiterbarkeit bietet. Weiterhin ein Backend, welches auf der einen Seite den einfachen Import von XML-Dateien in die geschaffene Datenstruktur erm�glicht und auf der anderen Seite diese mit weitreichenden M�glichkeiten der Individualisierung in eine dreidimensionale Ansicht umformen kann. Im dritten Teil, dem Frontend, k�nnen die Daten visuell und textuell betrachtet werden. Es ist also ein funktionsf�higes Framework entstanden.

\subsubsection{Prototypische Implementierung}
\label{sec:PrototypischeImplementierung}

Mit Hilfe des erstellten Klassenger�sts war es ohne hohen Implementierungsaufwand m�glich, die vom Betreuer dieser Arbeit zur Verf�gung gestellten Daten eines Gefechtssimulatorproxy darzustellen. Entstanden ist dabei ein Prototyp, der in der Lage ist, Datens�tze mit ihren Eigenschaften anzuzeigen.

\subsubsection{Beseitigung von Redundanzen durch Visualisierung}
\label{sec:BeseitigungVonRedundanzenDurchVisualisierung}

Fusion auf Basis von an einer Stelle befindlichen Datens�tzen geschieht im Prototyp automatisch, da sie visuell nicht unterscheidbar sind und somit als ein Datum zu erkennen sind.

Weiterhin wurde durch die Darstellung von Bewegungskegeln eine M�glichkeit der Reichweitenanalyse geschaffen. Somit ist es m�glich, auch Sichtungen, die zu unterschiedlichen Zeiten an unterschiedlichen Orten gemacht wurden, einem Objekt zuzuordnen. Zusammenfassend ist somit im Prototyp ein m�chtiges Werkzeug zur Visualisierung und visuellen Fusion von Daten entstanden.

\subsubsection{Plattformunabh�ngigkeit}
\label{sec:fazitPlattformunabhaengigkeit}

Aufgrund der Umsetzung in Java, der Verwendung von Swing und einer Grafikklassenbibliothek, die f�r alle g�ngigen Betriebssysteme vorhanden ist, entstand ein plattformunabh�ngiges System. Der Prototyp wurde erfolgreich auf Microsoft-Systemen (Windows XP, Vista, 7), auf Linux-Systemen (Ubuntu 9.04/9.10) und MacOS X getestet. Beschr�nkt wird die Unabh�ngigkeit nur von der Verf�gbarkeit nativer LWJGL-Bibliotheken und der OpenGL-F�higkeit des Betriebssystems. Da letztere f�r Windowssysteme bereits ab Win98 der Fall ist, scheitert es daran sicher mit sehr geringer Wahrscheinlichkeit.

\section{Weiterf�hrende Arbeit}
\section{Fazit}