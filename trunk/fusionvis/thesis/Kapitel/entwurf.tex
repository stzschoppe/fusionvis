%----------------------------------------------------------------%
% entwurf.tex						    			    										       %
%----------------------------------------------------------------%
\chapter{Design und Implementierung eines Visualisierungsprototypen}
In diesem Kapitel werden der Entwurf und die Implementierung eines Visualisierungsframeworks beschrieben. Dazu werden die einzelnen Komponenten des Frameworks beschrieben und ihr Zusammenspiel erl�utert. Basierend auf dem Wissen �ber die Bestandteile k�nnen die M�glichkeiten der Erweiterbarkeit und der Individualisierbarkeit skizziert werde. Die Grundlage hierf�r stellt zum einen eine prototypische Implementierung einer Visualisierung einer milit�rischen Lage und des weiteren der Versuch, Daten des NDP [ABK] dreidimensional darzustellen.

Abschlie�end werden interessante Aspekte der Implementierung aufgezeigt, die neben den umgesetzten Implementierungsentscheidungen auch andere Wege aufzeigen sollen.

Dem Leser, der sich vorrangig f�r den entstandenen Prototyp interessiert, dem sei der Abschnitt [Verweis] empfohlen. Die Details des Frameworks sollten f�r das Verst�ndnis des Funktionsumfangs und die Bedienung keine Rolle spielen, k�nnen aber bei Bedarf nachgeschlagen werden.

\section{Design eines Visualisierungsframeworks}
Das Ergebnis dieser Arbeit soll nicht nur eine potentielle Visualisierungs-Umgebung sein, sonder ein Framework, das zum einen die M�glichkeit bietet, mit geringem Aufwand Daten anzeigen zu k�nnen und auf der anderen Seite aber umfangreiche Erweiterungsm�glichkeiten zul�sst.
Wie das Framework konkret entworfen und umgesetzt wurde, soll im Folgenden dargestellt werden.

\subsection{Eingabedaten und Datenmodell}
Grundlage f�r die weitere Arbeit sollen Eingabedaten sein, die bestimmte Voraussetzungen erf�llen. Diese gestellten Bedingungen werden im Folgenden kurz beschrieben. Darauf basierend wird das Datenmodell erl�utert, in das die Quelldaten �berf�hrt werden sollen.

\subsubsection{Eingabedaten}
Um Daten dreidimensional visualisieren zu k�nnen, m�ssen diese bestimmte Voraussetzungen erf�llen. Diese sollen hier kurz aufgezeigt werden.

\paragraph{Identifizierbarkeit}
Unabh�ngig von der Art der Visualisierung ist es unabdingbar, dass jedes einzelne Datum identifizierbar ist. Aus diesem Grund sollte in den Quelldaten bereits eine eindeutige Benennung vorliegen. Sollte das nicht der Fall sein, muss dieser Misstand beim Importieren der Daten sp�testens behoben werden. 
Probleme, die sich ergeben k�nnen, wenn diese Bedingung nicht beachtet wird, sind zum einen, dass keine aussagekr�ftigen Berechnungen auf den importierten Daten durchgef�hrt werden k�nnen. Auch ist eine Markierung eines gezielten  Datensatzes unm�glich.

\paragraph{Lokalisierbarkeit}
Aus dem Ziel, die Eingabedaten im dreidimensionalen, kartesischen Raum darzustellen ergibt sich eine ganz logische Voraussetzung: Es sollte mindestens f�r jede der drei Dimensionen eine Eigenschaft der Daten existieren, die eine Positionierbarkeit m�glich macht. Zwar ist es genauso m�glich, die Daten auf einer Linie anzuordnen und somit nur eine Positionierungseigenschaft vorauszusetzen oder sich analog auf den zweidimensionalen Raum zu beschr�nken. Der daraus resultierende Informationsverlust muss aber in Kauf genommen werden.
Die Voraussetzungen an den Typ der f�r die Lokalisierung herangezogenen Eigenschaften sind nicht sehr streng. Zwar ist es zielf�hrend, wenn es sich hier um kontinuierliche oder zumindest diskret ganzzahlige  Gr��en handelt. Ist dies nicht der Fall, so m�ssen diese lediglich zur Berechnung der Anzeigekoordinaten mit einer geeigneten Abbildungsvorschrift umgerechnet werden.

Beispiele f�r unmittelbar geeignete Gr��en:
\begin{itemize}
	\item Entfernungsangaben 
  \item Ortsangaben in L�ngen- und Breitengrad
  \item Zeitangaben
\end{itemize}

Beispiele f�r mittelbar geeignete Gr��en:
\begin{itemize}
	\item IP [ABK] Adressen
  \item MAC [ABK] Adressen
  \item Zeichenketten allgemein
\end{itemize}

\paragraph{Weitere Eigenschaften}

\subsubsection{Datenmodell}

\subsection{�bersicht des Frameworks}

\subsection{Beschreibung der Komponenten}
\subsubsection{Importer}
\paragraph{Anbindung an XML}
\paragraph{Instanzieren des Datenmodells}
\subsubsection{Mapper}
\subsubsection{Viewer}


\subsection{Zusammenspiel der Komponenten}
\subsubsection{Visualisierungsprozess}
\subsubsection{Importer und Mapper}
\subsubsection{Viewer und Importer}
\subsubsection{Viewer und Mapper}

\section{Vorstellung der prototypischen Implementierung (Visualisierung einer milit�rischen Lage)}
\section{Erweiterbarkeit und Individualisierbarkeit}


\section{Fallstricke und interessante Aspekte der Implementierung}
\subsection{Mapper}
\subsubsection{Streckung der Eingabedaten auf den Projektionsbereich}
\subsubsection{Entfernungsberechnung mithilfe der Haversine Formel}
\subsubsection{Informationsverlust durch Projektion eines Kugelabschnitts auf eine Ebene}
\subsection{Viewer}
\subsubsection{Mousepicking}
\subsubsection{Bewegungskegel}


