%----------------------------------------------------------------%
% theorie.tex																			       %
%----------------------------------------------------------------%

\chapter{Multisensorische Daten und ihre Visualisierung}
\label{ch:theorie}

In diesem Kapitel soll zuerst der Zusammenhang zwischen dem Thema dieser Arbeit und den daf�r relevanten Theoretischen Grundlagen dargelegt werden. Darauf folgend sollen M�glichkeiten der Datenvisualisierung aufgezeigt. Einige davon sind in der prototypischen Implementierung umgesetzt, andere wurden nicht aufgenommen. Die gr�nde daf�r sollen an geeigneter Stelle dargelegt werden.

\section{Theoretische Einordnung}
\label{sec:TheoretischeEinordnung}
In diesem Abschnitt soll eine Einordnung der Arbeit in den theoretischen Kontext erfolgen. [TODO]

\subsection{Multisensorische Daten}
\label{sec:EigenschaftenVonMultisensorischenDaten}
Der Begriff der \emph{multisensorischen Daten} existiert in der Literatur nicht losgel�st. Interessant ist er nur in Verbindung der Fusion von Daten. Trotzdem soll an dieser Stelle eine kurze begriffliche Betrachtung stehen.

\label{beispiel}
Ein kleines Beispiel verdeutlicht sehr gut, worum es hier geht: Man stelle sich einen Tisch vor, auf dem zwei brennende Kerzen stehen. Davor befinden sich drei Beobachter. Jeder soll unabh�ngig von den anderen auf einem Blatt Papier jeden Gegenstand beschreiben, den er sieht. Bei den entstehenden Dokumenten handelt es sich um eine Form von multisensorischen Daten. Ein Datum ist dabei eine Beschreibung eines Gegenstandes. Ein Sensor ist jeweils einer der Beobachter.  Dies ist ein Ansatz, sich der Thematik zu n�hern. 

Ein anderer Ansatz k�nnte bei selbem Aufbau mit einer Person gemacht werden. Denn der Mensch, oder besser seine Sinne liefert schon multisensorische Daten. Die Person sieht eine Lichtquelle und f�hlt eine Hitzequelle. Diese beiden Beobachtungen sind die Daten, die Sensoren sind Tastsinn und Sehsinn.

Multisensorische Daten wie oben geschildert sind aber in der Theorie und Praxis nicht von Bedeutung, wenn sie f�r sich genommen werden. Schl�ssige Informationen daraus sind n�mlich nicht m�glich, ohne die Eingangsdaten zu interpretieren. Dieser Vorgang, man spricht von \emph{Multisensor Data Fusion}, soll in Abschnitt [REF] erl�utert werden.
\paragraph{Vorteile}
\label{sec:Vorteile}
In \cite{MDFLLINAS} werden neun Vorteile von multisensorischen Systemen (und den von ihnen gelieferten Daten) aufgez�hlt, die hier in Auswahl sinngem�� wiedergegeben werden sollen:

\begin{enumerate}
	\item \textbf{Robustheit:} Informationen k�nnen weiter gesammelt werden, auch wenn einer oder mehrere Sensoren keine Daten mehr liefern.

	\item \textbf{R\protect\"aumlich weitreichende Abdeckung:} Daten, die vielleicht nur von einem der vielen Sensoren gesehen werden, w�rden nicht entdeckt, wenn nur ein einzelner Sensor an der falschen Stelle steht.

	\item \textbf{Zeitlich weitreichende Abdeckung:} Nicht jeder Sensor steht zu jeder Zeit zur Verf�gung. Bei zeitlichem Versatz und �berlappung kann eine dauerhafte Datensammlung betrieben werden.

	\item \textbf{Gesteigerte Informationssicherheit:} Wird ein Datum durch mehrere Sensoren erfasst, verringert sich die Wahrscheinlichkeit eines Messfehlers.

	\item \textbf{Erh�hte Informationsdichte:} Die Aufl�sung, mit der mehrer Sensoren ein gebiet abtasten k�nnen, ist h�her als die Aufl�sung eines einzelnen Sensors.
\end{enumerate}

Diese Reihe von Vorteilen zeigt, dass die Datensammlung mit mehreren Sensoren durchaus betrieben werden sollte. Die entstehenden Daten bieten vor allem durch ihre Redundanzen  mehr und sichere Informationen als die eines einzelnen Sensors.

\paragraph{Nachteile}
\label{sec:Nachteile}
Die oben angesprochene Redundanz ist zugleich der gr��te Nachteil von multisensorischen Daten. Wie sich das �u�ert, sei noch mal an dem in Abschnitt \ref{beispiel} eingef�hrten Beispiel erl�utert: Die schriftlich gesammelten Daten der drei Beobachter werden an eine Person weitergegeben, die sich zum Zeitpunkt der Beobachtungen an einem anderen Ort aufhielt. Diese Person soll aus den Daten schlie�en, was sich auf dem Tisch befand. Sie hat dabei keine Informationen, welche Daten von welchem Beobachter stammen. Ebenfalls ist unbekannt, wie viele Beobachter �berhaupt beteiligt waren.

Aus den Daten k�nnte die auswertende Person schlie�en, dass sechs Kerzen auf dem Tisch befanden. Dies ist ein v�llig falsches Bild und der Grund f�r die Fehlinterpretation ist die Redundanz der Daten, die nicht beseitigt wurde. Nur wenn Datens�tze, die zu einem realen Objekt geh�ren, zusammengef�hrt werden, kann man die Vorteile von multisensorischen Daten nutzen.


 
\subsection{Multisensor-Fusion}
\label{sec:MultisensorFusion}
\cite{fusionintroduction} \cite{multisensorDataFusion} \cite{mathfusion}

\paragraph{Begriffsbestimmung}
\label{sec:Begriffsbestimmung}

\paragraph{Fusion Level}
\label{sec:FusionLevel}

\paragraph{Einordnung der Arbeit}
\label{sec:EinordnungDerArbeit}




\section{Visualisierungsm�glichkeiten}
\label{sec:Visualisierungsmoeglichkeiten}
 
\subsection{Darstellung der Dimensionen}
\label{sec:DarstellungDerDimensionen}

\subsubsection{Dreidimensionale Darstellung}
\label{sec:DreidimensionaleDarstellung}

\subsubsection{Vierdimensionale Darstellung}
\label{sec:VierdimensionaleDarstellung}

\subsubsection{Niederdimensionale  Ans�tze}
\label{sec:NiederdimensionaleAnsaetze}

\subsection{Darstellung der Datens�tze}
\label{ch:visualisierungDatum}

\subsubsection{Formen}
\label{sec:Formen}

\subsubsection{Visualisierung von Eigenschaften}
\label{sec:VisualisierungVonEigenschaften}

\subsection{Informationsgewinnung durch Visualisierung}
\label{sec:InformationsgewinnungDurchVisualisierung}

\subsubsection{Bewegungskegel}
\label{sec:TheorieBewegungskegel}

\subsubsection{Zugeh�rigkeitslinien}
\label{sec:Zugehoerigkeitslinien}
