\addcontentsline{toc}{chapter}{Anhang}
\chapter{Inhalt der CD}
Die beiligende CD enth�lt alle Daten und Programme um den implementierten Prototyp ausf�hren zu k�nnen. Zus�tzlich ist auf ihr eine digitale Version dieser Arbeit zu finden.

Der aktuellste Stand mit Entwicklungen, die �ber diese Arbeit hinaus gehen, sind auf der Seite \url{fusionvis.googlecode.com} im Downloadbereich und im SVN-Repository unter \url{fusionvis.googlecode.com/svn/trunk}.
\subsubsection{Inhalt:}

\begin{center}
\begin{longtable}{|l|l|l|} \hline
	Datei 										& Format 	& Bemerkungen 														\\ \hline
	Bachelorarbeit						& pdf			& Eine digitale Kopie der Bachelor Arbeit 		\\ \hline
	Java JDK 6								& exe			& Offline-Installation des JDK 6 Update 16 \\ \hline
	Java JRE 6								& exe			& Offline-Installation des JRE 6 Update 16 \\ \hline
	Eclipse Galileo						& zip			& Eclipse 3.5 											\\ \hline
	FusionVis									& Project	&	Java Source des Prototyps und des Frameworks			\\ \hline
	javadoc										& zip			&	JavaDoc Dokumentation											\\ \hline
	BattleSimVis							&	jar			&	Archiv mit den .class-Dateien des Prototyp \\ \hline
	BattleSimVis							& bat			& Datei zum Starten des Prototyps (Windows) \\ \hline
	BattleSimVis							& sh			&	Datei zum Starten des Prototyps (Linux)\\ \hline
	BattleSimVis\_lib					& Ordner	&	Native Grafikbibliotheken f�r Windows und Linux\\ \hline
	res												&	Ordner	& Testdatens�tze im XML-Format\\ \hline
\end{longtable}
\end{center}
